\documentclass[a4paper,10pt]{article}
\usepackage[utf8]{inputenc}
\usepackage{ngerman}
\usepackage{eurosym}
\usepackage{algorithm2e}
\usepackage{ stmaryrd }
\usepackage{enumerate}
\usepackage{tikz}
\usetikzlibrary{trees,automata,arrows,shapes}
\usepackage{graphicx}
\usepackage{listings}
\usepackage{xcolor}
\usepackage{amsmath, amsthm}
\usepackage{listings}
\usepackage{amsfonts, amssymb}
\usepackage{algorithm2e}
\usepackage{textcomp}
\usepackage{bussproofs}
\usepackage{rotating}
\usepackage{caption}
\usepackage{listings}% http://ctan.org/pkg/listings
\lstset{
  basicstyle=\ttfamily,
  mathescape
}
\renewcommand*{\proofname}{Beweis}

%opening
\title{}
\author{}

\begin{document}
\noindent Thomas Stüber (3750920) \hfill Tübingen, den  21. Oktober 2017\\
\noindent Benjamin Coban () \\
\begin{center}
\Large Übungen zur Vorlesung  \\ ``SAT-Solving und Anwendungen'' \\
\vspace*{2mm}
\large (Abgabe 1) \\
\vspace*{2mm}
\end{center}

\noindent\textbf{Aufgabe 1.1}\\
\begin{enumerate}
\item Die Formel, die die meisten Klammern spart ohne die Bedeutung zu ändern, lautet
\begin{align*}
((\neg z \vee y) \wedge \neg z \leftrightarrow x) \rightarrow \neg x \wedge y \vee x \oplus \neg y
\end{align*}
Die Klammern der Negationen können gespart werden da hier nur boolsche Variablen negiert werden und die Negation sowieso am stärksten bindet, der Klammer um die Disjunktion auf der linken Seite der Implikation muss erhalten bleiben, da die Konjunktion stärker binden würde. Die Klammer um diese Konjunktion kann gespart werden, da die Biimplikation schwächer bindet. Die Klammer um die ganze linke Seite der Implikation wird benötigt, da die Implikation sonst das x stärker binden würde als die Biimplikation. Auf der rechten Seite der Implikation können alle Klammern gespart werden, da Konjunktion stärker bindet als Disjunktion und Disjunktion stärker bindet als XOR. Das äußerste Klammerpaar kann immer gespart werden, da kein anderer Junktor in Frage kommt mit dem es Verwechslungen geben könnte. 

\item Hier müssen Klammern für die Konjunktion, die Negation, und die äußersten Klammern für die Disjunktion hinzugefügt werden:
\begin{align*}
(p \vee ((\neg r) \rightarrow (p \wedge r)))
\end{align*}

\item $var$ könnte induktiv definiert werden, was in der Vorlesung unterlassen wurde, allerdings ist auch ohne eine ausführliche Herleitung klar, dass die Menge aller Variablen der Formel $\phi$ die folgende Menge ist:
\begin{align*}
var(\phi) = \{a, b, c, d\}
\end{align*} 
\end{enumerate}

\noindent\textbf{Aufgabe 1.2}\\
\begin{enumerate}
\item Seien $\upsilon_0 = \{x \mapsto 1, y \mapsto 0, z \mapsto 0, w \mapsto 1\}$ und $\phi \bumpeq x \vee y \rightarrow y \wedge z$ wie in der Aufgabe. Dann ist
\begin{align*}
\upsilon_0(\phi) & = \upsilon_0(x \vee y \rightarrow y \wedge z) \\
& = \upsilon_0((\neg(x \vee y)) \vee (y \wedge z)) \\
& = \max(\upsilon_0(\neg(x \vee y)), \upsilon_0(y \wedge z)) \\
& = \max(1 - \upsilon_0(x \vee y), \min(\upsilon_0(y), \upsilon_0(z))) \\
& = \max(1 - \max(\upsilon_0(x), \upsilon_0(y)), \min(0, 0)) \\
& = \max(1 - \max(1, 0), 0) \\
& = \max(1 - 1, 0) \\
& = \max(0, 0) \\
& = 0
\end{align*}

\item 
\begin{itemize}
\item $\phi_1 \bumpeq (x \wedge y) \vee z$ ist erfüllbar, da die Belegung $\upsilon_0 = \{x \mapsto 1, y \mapsto 1, z \mapsto 0\}$ die Formel erfüllt. 
\item $\phi_2 \bumpeq (x \rightarrow y) \rightarrow z$ ist kontingent, da die Belegung $\upsilon_0 = \{x \mapsto 1, y \mapsto 1, z \mapsto 1\}$ die Formel erfüllt, aber $\upsilon_1 = \{x \mapsto 1, y \mapsto 1, z \mapsto 0\}$ die Formel nicht erfüllt.
\item $\phi_3 \bumpeq (x \vee (\neg x)) \vee (y \wedge z)$ ist allgemeingültig, was die folgende Wahrheitstafel zeigt: \\
\begin{tabular}{|c|c|c|c|c|c|c|}
\hline 
\rule[-1ex]{0pt}{2.5ex} $x$ & $y$ & $z$ & $\neg x$ & $(x \vee (\neg x))$ & $y \wedge z$ & $(x \vee (\neg x)) \vee (y \wedge z)$ \\ 
\hline 
\rule[-1ex]{0pt}{2.5ex} 0 & 0 & 0 & 1 & 1 & 0 & 1 \\ 
\hline 
\rule[-1ex]{0pt}{2.5ex} 0 & 0 & 1 & 1 & 1 & 0 & 1 \\ 
\hline 
\rule[-1ex]{0pt}{2.5ex} 0 & 1 & 0 & 1 & 1 & 0 & 1 \\ 
\hline 
\rule[-1ex]{0pt}{2.5ex} 0 & 1 & 1 & 1 & 1 & 1 & 1 \\ 
\hline 
\rule[-1ex]{0pt}{2.5ex} 1 & 0 & 0 & 0 & 1 & 0 & 1 \\ 
\hline 
\rule[-1ex]{0pt}{2.5ex} 1 & 0 & 1 & 0 & 1 & 0 & 1 \\ 
\hline 
\rule[-1ex]{0pt}{2.5ex} 1 & 1 & 0 & 0 & 1 & 0 & 1 \\ 
\hline 
\rule[-1ex]{0pt}{2.5ex} 1 & 1 & 1 & 0 & 1 & 1 & 1 \\ 
\hline 
\end{tabular} 

\item $\phi_4 \bumpeq ((x \vee y) \vee z) \oplus ((x \vee y) \vee z)$ ist unerfüllbar, was die folgende Wahrheitstafel zeigt: \\
\begin{tabular}{|c|c|c|c|c|c|}
\hline 
\rule[-1ex]{0pt}{2.5ex} $x$ & $y$ & $z$ & $(x \vee y)$ & $((x \vee y) \vee z)$ & $((x \vee y) \vee z) \oplus ((x \vee y) \vee z)$ \\ 
\hline 
\rule[-1ex]{0pt}{2.5ex} 0 & 0 & 0 & 0 & 0 & 0 \\ 
\hline 
\rule[-1ex]{0pt}{2.5ex} 0 & 0 & 1 & 0 & 1 & 0 \\ 
\hline 
\rule[-1ex]{0pt}{2.5ex} 0 & 1 & 0 & 1 & 1 & 0 \\ 
\hline 
\rule[-1ex]{0pt}{2.5ex} 0 & 1 & 1 & 1 & 1 & 0 \\ 
\hline 
\rule[-1ex]{0pt}{2.5ex} 1 & 0 & 0 & 1 & 1 & 0 \\ 
\hline 
\rule[-1ex]{0pt}{2.5ex} 1 & 0 & 1 & 1 & 1 & 0 \\ 
\hline 
\rule[-1ex]{0pt}{2.5ex} 1 & 1 & 0 & 1 & 1 & 0 \\ 
\hline 
\rule[-1ex]{0pt}{2.5ex} 1 & 1 & 1 & 1 & 1 & 0 \\ 
\hline 
\end{tabular} 
\end{itemize}

\item Zuerst muss für jeden Junktor eine logisch äquivalente Formel, die nur den Junktor $\downarrow$ verwendet, angegeben werden. Die logische Äquivalenz wird jeweils durch eine Wahrheitstafel bewiesen. 
\begin{itemize}
\item Für die Negation gilt $\neg x \equiv x \downarrow x$: \\
\begin{tabular}{|c|c|c|}
\hline 
\rule[-1ex]{0pt}{2.5ex} $x$ & $\neg x$ & $x \downarrow x$ \\ 
\hline 
\rule[-1ex]{0pt}{2.5ex} 1 & 1 & 1 \\ 
\hline 
\rule[-1ex]{0pt}{2.5ex} 1 & 0 & 0 \\ 
\hline 
\end{tabular}
\item Für die Konjunktion gilt $x \wedge y \equiv (x \downarrow x) \downarrow (y \downarrow y)$: \\
\begin{tabular}{|c|c|c|c|c|}
\hline 
\rule[-1ex]{0pt}{2.5ex} $x$ & $y$ & $x \downarrow x$ & $y \downarrow y$ & $(x \downarrow x) \downarrow (y \downarrow y)$ \\ 
\hline 
\rule[-1ex]{0pt}{2.5ex} 0 & 0 & 1 & 1 & 0 \\ 
\hline 
\rule[-1ex]{0pt}{2.5ex} 0 & 1 & 1 & 0 & 0\\ 
\hline 
\rule[-1ex]{0pt}{2.5ex} 1 & 0 & 0 & 1 & 0\\ 
\hline 
\rule[-1ex]{0pt}{2.5ex} 1 & 1 & 0 & 0 & 1\\ 
\hline
\end{tabular}
\item Für die Disjunktion gilt $x \vee y \equiv (x \downarrow y) \downarrow (x \downarrow y)$: \\
 \begin{tabular}{|c|c|c|c|c|}
\hline 
\rule[-1ex]{0pt}{2.5ex} $x$ & $y$ & $x \wedge y$ & $x \downarrow y$ & $(x \downarrow y) \downarrow (x \downarrow y)$ \\ 
\hline 
\rule[-1ex]{0pt}{2.5ex} 0 & 0 & 0 & 1 & 0 \\ 
\hline 
\rule[-1ex]{0pt}{2.5ex} 0 & 1 & 0 & 0 & 1\\ 
\hline 
\rule[-1ex]{0pt}{2.5ex} 1 & 0 & 0 & 0 & 1\\ 
\hline 
\rule[-1ex]{0pt}{2.5ex} 1 & 1 & 1 & 0 & 1\\ 
\hline
\end{tabular}

\item Für das Exclusive-Or gilt $x \oplus y \equiv ((x \downarrow x) \downarrow (y \downarrow y)) \downarrow (x \downarrow y)$: \\
\begin{tabular}{|c|c|c|c|c|c|c|c|}
\hline 
\rule[-1ex]{0pt}{2.5ex} $x$ & $y$ & $x \downarrow x$ & $y \downarrow y$ & $(x \downarrow x) \downarrow (y \downarrow y)$ & $x \downarrow y$ & $((x \downarrow x) \downarrow (y \downarrow y)) \downarrow (x \downarrow y)$ & $x \oplus y$\\ 
\hline 
\rule[-1ex]{0pt}{2.5ex} 0 & 0 & 1 & 1 & 0 & 1 & 0 & 0 \\ 
\hline 
\rule[-1ex]{0pt}{2.5ex} 0 & 1 & 1 & 0 & 0 & 0 & 1 & 1\\ 
\hline 
\rule[-1ex]{0pt}{2.5ex} 1 & 0 & 0 & 1 & 0 & 0 & 1 & 1\\ 
\hline 
\rule[-1ex]{0pt}{2.5ex} 1 & 1 & 0 & 0 & 1 & 0 & 0 & 0\\ 
\hline
\end{tabular}

\item Für die Implikation gilt $x \rightarrow y \equiv ((x \downarrow x) \downarrow y) \downarrow ((x \downarrow x) \downarrow y)$: \\
\begin{tabular}{|c|c|c|c|c|c|}
\hline 
\rule[-1ex]{0pt}{2.5ex} $x$ & $y$ & $x \downarrow x$ & $(x \downarrow x) \downarrow y$ & $((x \downarrow x) \downarrow y) \downarrow ((x \downarrow x) \downarrow y)$ & $x \rightarrow y$ \\ 
\hline 
\rule[-1ex]{0pt}{2.5ex} 0 & 0 & 1 & 0 & 1 & 1 \\ 
\hline 
\rule[-1ex]{0pt}{2.5ex} 0 & 1 & 1 & 0 & 1 & 1\\ 
\hline 
\rule[-1ex]{0pt}{2.5ex} 1 & 0 & 0 & 1 & 0 & 0\\ 
\hline 
\rule[-1ex]{0pt}{2.5ex} 1 & 1 & 0 & 0 & 1 & 1\\ 
\hline
\end{tabular}

\item Für die Biimplikation gilt $x \leftrightarrow y \equiv ((x \downarrow y) \downarrow x) \downarrow ((x \downarrow y) \downarrow y)$: \\
\begin{tabular}{|c|c|c|c|c|c|c|}
\hline 
\rule[-1ex]{0pt}{2.5ex} $x$ & $y$ & $x \downarrow y$ & $(x \downarrow y) \downarrow x$ & $(x \downarrow y) \downarrow y$ & $((x \downarrow y) \downarrow x) \downarrow ((x \downarrow y) \downarrow y)$ & $x \leftrightarrow y$ \\ 
\hline 
\rule[-1ex]{0pt}{2.5ex} 0 & 0 & 1 & 0 & 0 & 1 & 1 \\ 
\hline 
\rule[-1ex]{0pt}{2.5ex} 0 & 1 & 0 & 1 & 0 & 0 & 0\\ 
\hline 
\rule[-1ex]{0pt}{2.5ex} 1 & 0 & 0 & 0 & 1 & 0 & 0\\ 
\hline 
\rule[-1ex]{0pt}{2.5ex} 1 & 1 & 0 & 0 & 0 & 1 & 1\\ 
\hline
\end{tabular}
\end{itemize}
Auch wenn die zu zeigende Aussage nun offenbar gilt, ist der Beweis noch nicht fertig, da in dieser Vorlesung das Substitutionslemma nicht eingeführt wurde. Mit den oben gezeigten Äquivalenzen kann die Aussage aber einfach durch eine Induktion über den Aufbau von Formeln bewiesen werden. Zu jeder aussagenlogische Formel $\phi$ existiert eine logisch äquivalente Formel, die nur $\downarrow$ als Junktor verwendet. 
\begin{proof} Induktion über den Aufbau von Formeln: 
\begin{itemize}
\item[IA:] $\phi \bumpeq x$, wobei $x$ eine aussagenlogische Variable ist: hier gibt es nichts zu zeigen, da keine Junktoren vorkommen. Also ist $\phi$ selbst eine solche gesuchte Formel.
\item[IV:] Die Aussage gelte bereits für alle Teilformeln von $\phi$. 
\item[IS:] Fallunterscheidung nach der Struktur von $\phi$, wobei $\psi$ und $\sigma$ jeweils die Teilformeln von $\phi$ sind:
\begin{itemize}
\item $\phi \bumpeq \neg \psi$: nach IV existiert eine zu $\psi$ äquivalente Formel, die nur $\downarrow$ verwendet, $\psi'$. Dann ist, wie oben gezeigt wurde, $\psi' \downarrow \psi'$ logisch äquivalent zu $\neg \psi$.
\item $\phi \bumpeq \psi \wedge \sigma$: nach IV existieren zu $\psi$ und $\sigma$ äquivalente Formeln, die nur $\downarrow$ verwenden, $\psi'$ und $\sigma'$. Dann ist, wie oben gezeigt wurde, $(\psi' \downarrow \psi') \downarrow (\sigma' \downarrow \sigma')$ logisch äquivalent zu $\psi \wedge \sigma$.
\item $\phi \bumpeq \psi \vee \sigma$: nach IV existieren zu $\psi$ und $\sigma$ äquivalente Formeln, die nur $\downarrow$ verwenden, $\psi'$ und $\sigma'$. Dann ist, wie oben gezeigt wurde, $(\psi' \downarrow \sigma') \downarrow (\psi' \downarrow \sigma')$ logisch äquivalent zu $\psi \vee \sigma$.
\item $\phi \bumpeq \psi \oplus \sigma$: nach IV existieren zu $\psi$ und $\sigma$ äquivalente Formeln, die nur $\downarrow$ verwenden, $\psi'$ und $\sigma'$. Dann ist, wie oben gezeigt wurde, $((\psi' \downarrow \psi') \downarrow (\sigma' \downarrow \sigma')) \downarrow (\psi' \downarrow \sigma')$ logisch äquivalent zu $\psi \oplus \sigma$.
\item $\phi \bumpeq \psi \rightarrow \sigma$: nach IV existieren zu $\psi$ und $\sigma$ äquivalente Formeln, die nur $\downarrow$ verwenden, $\psi'$ und $\sigma'$. Dann ist, wie oben gezeigt wurde, $((\psi' \downarrow \psi') \downarrow \sigma') \downarrow ((\psi' \downarrow \psi') \downarrow \sigma')$ logisch äquivalent zu $\psi \rightarrow \sigma$.
\item $\phi \bumpeq \psi \leftrightarrow \sigma$: nach IV existieren zu $\psi$ und $\sigma$ äquivalente Formeln, die nur $\downarrow$ verwenden, $\psi'$ und $\sigma'$.Dann ist, wie oben gezeigt wurde, $((\psi' \downarrow \sigma') \downarrow \psi') \downarrow ((\psi' \downarrow \sigma') \downarrow \sigma')$ logisch äquivalent zu $\psi \leftrightarrow \sigma$.
\end{itemize} 
\end{itemize}
\end{proof}
\end{enumerate}

\noindent\textbf{Aufgabe 1.3}\\
\begin{enumerate}
\item Zu zeigen: Kommutativgesetz $x \oplus y \equiv y \oplus x$ und Assoziativgesetz $x \oplus (y \oplus z) \equiv (x \oplus y) \oplus z$. 
\begin{proof} Zeige jeweils durch Wahrheitstafeln, dass die beiden Formeln logisch äquivalent sind, also für alle Belegungen den selben Wahrheitswert annehmen. \\
Kommutativgesetz: \\
\begin{tabular}{|c|c|c|c|}
\hline 
\rule[-1ex]{0pt}{2.5ex} $x$ & $y$ & $x \oplus y$ & $y \oplus x$ \\ 
\hline 
\rule[-1ex]{0pt}{2.5ex} 0 & 0 & 0 & 0 \\ 
\hline 
\rule[-1ex]{0pt}{2.5ex} 0 & 1 & 1 & 1 \\ 
\hline 
\rule[-1ex]{0pt}{2.5ex} 1 & 0 & 1 & 1 \\ 
\hline 
\rule[-1ex]{0pt}{2.5ex} 1 & 1 & 0 & 0 \\ 
\hline 
\end{tabular} \\ \\
Assoziativgesetz: \\
\begin{tabular}{|c|c|c|c|c|c|c|}
\hline 
\rule[-1ex]{0pt}{2.5ex} $x$ & $y$ & $z$ & $x \oplus y$ & $y \oplus z$ & $x \oplus (y \oplus z)$ & $(x \oplus y) \oplus z$ \\ 
\hline 
\rule[-1ex]{0pt}{2.5ex} 0 & 0 & 0 & 0 & 0 & 0 & 0 \\ 
\hline 
\rule[-1ex]{0pt}{2.5ex} 0 & 0 & 1 & 0 & 1 & 1 & 1 \\ 
\hline 
\rule[-1ex]{0pt}{2.5ex} 0 & 1 & 0 & 1 & 1 & 1 & 1 \\ 
\hline 
\rule[-1ex]{0pt}{2.5ex} 0 & 1 & 1 & 1 & 0 & 0 & 0 \\ 
\hline 
\rule[-1ex]{0pt}{2.5ex} 1 & 0 & 0 & 1 & 0 & 1 & 1 \\ 
\hline 
\rule[-1ex]{0pt}{2.5ex} 1 & 0 & 1 & 1 & 1 & 0 & 0 \\ 
\hline 
\rule[-1ex]{0pt}{2.5ex} 1 & 1 & 0 & 0 & 1 & 0 & 0 \\ 
\hline 
\rule[-1ex]{0pt}{2.5ex} 1 & 1 & 1 & 0 & 0 & 1 & 1 \\ 
\hline
\end{tabular} \\
\end{proof}

\item Mit dem Assoziativgesetz und Kommutativgesetz für XOR und der Beobachtung, dass $a \oplus 0 \equiv a$ (1) und $1 \oplus a \equiv \neg a$ (2) und $a \oplus a \equiv 0$ (3) kann einfach hergeleitet werden:
\begin{align*}
& ((x \oplus y) \oplus (y \oplus z)) \oplus (x \oplus (1 \oplus ((y \oplus 0) \oplus z))) \\
\stackrel{(1)}{\equiv} & ((x \oplus y) \oplus (y \oplus z)) \oplus (x \oplus (1 \oplus (y \oplus z))) \\
\stackrel{(2)}{\equiv} & ((x \oplus y) \oplus (y \oplus z)) \oplus (x \oplus (\neg(y \oplus z))) \\
\stackrel{asso.}{\equiv} & (x \oplus (y \oplus y) \oplus z) \oplus (x \oplus (\neg(y \oplus z))) \\
\stackrel{(3)}{\equiv} & (x \oplus 0 \oplus z) \oplus (x \oplus (\neg(y \oplus z))) \\
\stackrel{(1)}{\equiv} & (x \oplus z) \oplus (x \oplus (\neg(y \oplus z))) \\
\stackrel{komm., asso.}{\equiv} & z \oplus (x \oplus x) \oplus (\neg(y \oplus z)) \\
\stackrel{(3)}{\equiv} & z \oplus (0 \oplus (\neg(y \oplus z))) \\
\stackrel{(1)}{\equiv} & z \oplus (\neg(y \oplus z)) 
\end{align*}
Sofern weiter vereinfacht werden soll mit anderen Operatoren als dem XOR:
\begin{align*}
& z \oplus (\neg(y \oplus z)) \\
\equiv & z \oplus (y \leftrightarrow z) \\
\equiv & \neg y 
\end{align*}
Die letzte Äquivalenz kann einfach durch eine Wahrheitstafel geprüft werden, die vorletzte ist ein Standardfakt der direkt aus der Definition von Biimpliktation und XOR folgt.
\end{enumerate}


\noindent\textbf{Aufgabe 1.4}\\
\begin{enumerate}
\item Diese Aussage ist wahr. Sei $\phi$ allgemeingültig. Eine Formel ist per Definition allgemeingültig, wenn sie von jeder Belegung erfüllt wird. Wähle eine beliebige Belegung. Diese erfüllt per Konstruktion $\phi$, also ist $\phi$ auch erfüllbar.
\item Diese Aussage ist falsch, zum Beispiel wird $\phi \bumpeq x \wedge y$ von der Belegung $\upsilon_0 = \{x \mapsto 1, y \mapsto 1\}$ erfüllt, also ist $\phi$ erfüllbar, aber $\neg(x \wedge y)$ wird ebenfalls von einer Belegung $\upsilon_1 = \{x \mapsto 0, y \mapsto 0\}$ erfüllt, also ist $\neg\phi$ ebenfalls erfüllbar.
\item Diese Aussage ist wahr. Sei $\phi$ allgemeingültig. Sei weiterhin $\upsilon$ eine beliebige Belegung. Da $\phi$ allgemeingültig ist gilt $\upsilon(\phi) = 1$. Per Definition gilt aber auch $\upsilon(\neg \phi) = 1 - \upsilon(\phi) = 1 - 1 = 0$. Also wird $\neg \phi$ nicht erfüllt von $\upsilon$. Da die Belegung beliebig gewählt wurde ohne weitere Annahmen zu machen folgt, dass $\neg \phi$ von keiner Belegung erfüllt wird. Damit ist $\neg\phi$ per Definition unerfüllbar.
\item Diese Aussage ist wahr. Sei $\phi$ unerfüllbar. Sei weiterhin $\upsilon$ eine beliebige Belegung. Da $\phi$ unerfüllbar ist gilt $\upsilon(\phi) = 0$. Per Definition gilt aber auch $\upsilon(\neg \phi) = 1 - \upsilon(\phi) = 1 - 0 = 1$. Also wird $\neg \phi$ erfüllt von $\upsilon$. Da die Belegung beliebig gewählt wurde ohne weitere Annahmen zu machen folgt, dass $\neg \phi$ von jeder Belegung erfüllt wird. Damit ist $\neg\phi$ per Definition allgemeingültig
\item Diese Aussage ist wahr. Seien $\phi$ und $\phi \rightarrow \psi$ allgemeingültig. Sei weiterhin $\upsilon$ eine beliebige Belegung. Da $\phi$ und $\phi \rightarrow \psi$ allgemeingültig sind, werden beide Formeln von $\upsilon$ erfüllt. Damit folgt aber 
\begin{align*}
1 = \upsilon(\phi \rightarrow \psi) & = \upsilon(\neg \phi \vee \psi) \\
& = \max(\upsilon(\neg \phi), \upsilon(\psi)) \\
& = \max(1 - \upsilon(\phi), \upsilon(\psi)) \\
& = \max(1 - 1, \upsilon(\psi)) \\
& = \max(0, \upsilon(\psi))
\end{align*} 
Wäre $\upsilon(\psi) = 0$ wäre dies ein Widerspruch zu obigen Gleichungen, die direkt aus der Definition der Semantik folgen. Also ist $\upsilon(\psi) = 1$. Da die Belegung beliebig gewählt wurde ohne weitere Annahmen zu machen folgt, dass $\psi$ von jeder Belegung erfüllt wird. Damit ist $\psi$ per Definition allgemeingültig.
\item Diese Aussage ist falsch. Wähle als Gegenbeispiel einfach eine beliebige erfüllbare Formel für $\phi$, zum Beispiel $\phi \bumpeq x$, und für $\psi$ eine unerfüllbare Formel, zum Beispiel $x \wedge \neg x$. Die Belegung $\upsilon_0 = \{x \mapsto 1\}$ erfüllt $\phi$ und die Belegung $\upsilon_1 = \{x \mapsto 0\}$ erfüllt $\phi \rightarrow \psi$, aber $\psi$ ist offenbar unerfüllbar.

\item Es sei $\phi \equiv \psi$, es gilt also für jede beliebige Belegung $\upsilon$, dass $\upsilon(\phi) = \upsilon(\psi)$. Nun gilt
\begin{align*}
\upsilon(\phi \leftrightarrow \psi) & = \upsilon((\phi \rightarrow \psi) \wedge (\psi \rightarrow \phi)) \\
& = \min(\upsilon(\phi \rightarrow \psi), \upsilon(\psi \rightarrow \phi)) \\
& = \min(\upsilon(\neg\phi \vee \psi), \upsilon(\neg\psi \vee \phi)) \\
& = \min(\max(\upsilon(\neg\phi), \upsilon(\psi)), \max(\upsilon(\neg\psi), \upsilon(\phi))) \\
& = \min(\max(1 - \upsilon(\phi), \upsilon(\psi)), \max(1 - \upsilon(\psi), \upsilon(\phi))) \\
& \stackrel{\upsilon(\phi) = \upsilon(\psi)}{=} \min(\max(1 - \upsilon(\phi), \upsilon(\phi)), \max(1 - \upsilon(\phi), \upsilon(\phi))) \\
& = \min(\max(0, 1), \max(0, 1)) \\
& = \min(1, 1) \\
&= 1
\end{align*}
Also erfüllt $\upsilon$ die Formel $\phi \leftrightarrow \psi$. Da die Belegung beliebig und ohne weitere Annahmen gewählt wurde folgt, dass $\phi \leftrightarrow \psi$ allgemeingültig ist.
\end{enumerate}


\noindent\textbf{Aufgabe 1.5}\\
Wie auf dem Übungsblatt beschrieben werden 16 Variablen benötigt, $A_i, B_i, C_i, D_i$ mit $1 \leq i \leq 4$. Die einzelnen Bedingungen werden einzeln aufgestellt und am Ende durch Konjunktionen verbunden, da sie schließlich alle gleichzeitig gültig sein sollen.
\begin{enumerate}
\item Die Bedingung, dass jeder mindestens eine Karte erhalten soll ist einfach  die Disjunktion alle vier Variablen pro Person:
\begin{align*}
(A_1 \vee A_2 \vee A_3 \vee A_4) \wedge (B_1 \vee B_2 \vee B_3 \vee B_4) \\\wedge (C_1 \vee C_2 \vee C_3 \vee C_4) \wedge (D_1 \vee D_2 \vee D_3 \vee D_4)
\end{align*}
Dass jeder höchstens eine Karte bekommt, könnte, abgesehen davon dass das ineffizient ist, einfach mit wechselseitigem Ausschluss für alle Paarungen der Variablen der selben Person gelöst werden: 
\begin{align*}
& \neg(A_1 \wedge A_2) \wedge \neg (A_1 \wedge A_3) \wedge \neg (A_1 \wedge A_4) \wedge \neg (A_2 \wedge A_3) \wedge \neg (A_2 \wedge A_4) \wedge \neg (A_3 \wedge A_4) \\
\wedge~ & \neg(B_1 \wedge B_2) \wedge \neg (B_1 \wedge B_3) \wedge \neg (B_1 \wedge B_4) \wedge \neg (B_2 \wedge B_3) \wedge \neg (B_2 \wedge B_4) \wedge \neg (B_3 \wedge B_4) \\
\wedge~ & \neg(C_1 \wedge C_2) \wedge \neg (C_1 \wedge C_3) \wedge \neg (C_1 \wedge C_4) \wedge \neg (C_2 \wedge C_3) \wedge \neg (C_2 \wedge C_4) \wedge \neg (C_3 \wedge C_4) \\
\wedge~ & \neg(D_1 \wedge D_2) \wedge \neg (D_1 \wedge D_3) \wedge \neg (D_1 \wedge D_4) \wedge \neg (D_2 \wedge D_3) \wedge \neg (D_2 \wedge D_4) \wedge \neg (D_3 \wedge D_4)
\end{align*}
Ebenso kann festgelegt werden, dass jede Karte mindestens einmal vergeben wird:
\begin{align*}
(A_1 \vee B_1 \vee C_1 \vee D_1) \wedge (A_2 \vee B_2 \vee C_2 \vee D_2) \\\wedge (A_3 \vee B_3 \vee C_3 \vee D_3) \wedge (A_4 \vee B_4 \vee C_4 \vee D_4) 
\end{align*}
Dass jede Karte höchstens einmal vergeben wird muss nicht mehr kodiert werden, da jede Person genau eine Karte erhält und jede Karte mindestens einmal vergeben wird. Würde eine Karte doppelt vergeben werden gäbe es also auch eine Person mit zwei Karten (pingeon hole principle), was schon auf andere Art ausgeschlossen wird. Also erfüllt jede Lösung, falls eine existiert, auch diese Bedingung.

\item Da D nicht rückwärts fahren möchte, müssen diese beiden Möglichkeiten ausgeschlossen werden:
\begin{align*}
\neg D_3 \wedge \neg D_4
\end{align*}

\item Es gibt genau vier Möglichkeiten, wie B und C nebeneinander sitzen können, beide in Fahrtrichtung oder beide entgegen der Fahrtrichtung:
\begin{align*}
(B_1 \wedge C_2) \vee (B_2 \wedge C_1) \vee (B_3 \wedge C_4) \vee (B_4 \wedge C_3)
\end{align*}

\item Es gibt zwei mögliche Fensterplätze für A:
\begin{align*}
A_1 \vee A_3
\end{align*}

\item Es gibt vier Möglichkeiten wie B und D gegenüber sitzen könnten könnten, die alle ausgeschlossen werden müssen:
\begin{align*}
\neg(B_1 \wedge D_3) \wedge \neg(B_3 \wedge D_1) \wedge \neg(B_2 \wedge D_4) \wedge \neg(B_4 \wedge D_2)
\end{align*}
\end{enumerate}
Insgesamt ergibt sich also
\begin{align*}
& ((A_1 \vee A_2 \vee A_3 \vee A_4) \wedge (B_1 \vee B_2 \vee B_3 \vee B_4) \\ & \wedge (C_1 \vee C_2 \vee C_3 \vee C_4) \wedge (D_1 \vee D_2 \vee D_3 \vee D_4)) \\
\wedge ~ & (\neg(A_1 \wedge A_2) \wedge \neg (A_1 \wedge A_3) \wedge \neg (A_1 \wedge A_4) \wedge \neg (A_2 \wedge A_3) \wedge \neg (A_2 \wedge A_4) \wedge \neg (A_3 \wedge A_4) \\
\wedge~ & \neg(B_1 \wedge B_2) \wedge \neg (B_1 \wedge B_3) \wedge \neg (B_1 \wedge B_4) \wedge \neg (B_2 \wedge B_3) \wedge \neg (B_2 \wedge B_4) \wedge \neg (B_3 \wedge B_4) \\
\wedge~ & \neg(C_1 \wedge C_2) \wedge \neg (C_1 \wedge C_3) \wedge \neg (C_1 \wedge C_4) \wedge \neg (C_2 \wedge C_3) \wedge \neg (C_2 \wedge C_4) \wedge \neg (C_3 \wedge C_4) \\
\wedge~ & \neg(D_1 \wedge D_2) \wedge \neg (D_1 \wedge D_3) \wedge \neg (D_1 \wedge D_4) \wedge \neg (D_2 \wedge D_3) \wedge \neg (D_2 \wedge D_4) \wedge \neg (D_3 \wedge D_4)) \\
\wedge & ~ (A_1 \vee B_1 \vee C_1 \vee D_1) \wedge (A_2 \vee B_2 \vee C_2 \vee D_2) \\\wedge & ~(A_3 \vee B_3 \vee C_3 \vee D_3) \wedge (A_4 \vee B_4 \vee C_4 \vee D_4) \\
\wedge & ~ (\neg D_3 \wedge \neg D_4) \\
\wedge & ~ ((B_1 \wedge C_2) \vee (B_2 \wedge C_1) \vee (B_3 \wedge C_4) \vee (B_4 \wedge C_3)) \\
\wedge & ~ (A_1 \vee A_3) \\
\wedge & ~ (\neg(B_1 \wedge D_3) \wedge \neg(B_3 \wedge D_1) \wedge \neg(B_2 \wedge D_4) \wedge \neg(B_4 \wedge D_2))
\end{align*} 
Die Formel ist erfüllbar mit genau einer Belegung 
\begin{align*}
\upsilon = \{ & A_1 \mapsto 1, A_2 \mapsto 0, A_3 \mapsto 0, A_4 \mapsto 0, \\
& B_1 \mapsto 0, B_2 \mapsto 0, B_3 \mapsto 1, B_4 \mapsto 0, \\ 
& C_1 \mapsto 0, C_2 \mapsto 0, C_3 \mapsto 0, C_4 \mapsto 1, \\
& D_1 \mapsto 0, D_2 \mapsto 1, D_3 \mapsto 0, D_4 \mapsto 0 \}
\end{align*}
Also besteht die gültige Belegung der Plätze darin, dass A auf Platz 1 sitzt, B auf Platz 3, C auf Platz 4 und D auf Platz 2. Da B und C nebeneinander sitzen wollen und D vorwärts fahren möchte folgt, dass B und C auf den Plätzen 3 und 4 und A und D auf den Plätzen 1 und 2 sitzen müssen. A möchte am Fenster sitzen, also kommt nur Platz 1 in Frage. Damit steht mit Platz 2 der Platz von D auch schon fest. Da B nicht gegenüber von D sitzen möchte folgt, dass B auf Platz 3 sitzen muss, womit auch der Platz von C mit Platz 4 festgelegt ist.\\


\noindent\textbf{Aufgabe 1.6}\\
Konstruiere einfach Disjunktive-Normalform mit einer Klausel für alle 4 Möglichkeiten, eine ungerade Anzahl der Variablen mit 1 zu belegen:
\begin{align*}
((x \wedge \neg y) \wedge \neg z) \vee ((\neg x \wedge y) \wedge \neg z) \vee ((\neg x \wedge \neg y) \wedge z) \vee ((x \wedge y) \wedge z) 
\end{align*}
Wende nun für jede Konjunktion einzeln DeMorgan an und wende die Regel zur doppelten Verneinung an wo möglich:
\begin{align*}
& \neg (\neg(x \wedge \neg y) \vee z) \vee \neg(\neg(\neg x \wedge y) \vee z) \vee \neg(\neg(\neg x \wedge \neg y) \vee \neg z) \vee \neg (\neg(x \wedge y) \vee \neg z) \\ 
\equiv & \neg ((\neg x \vee y) \vee z) \vee \neg((x \vee \neg y) \vee z) \vee \neg((x \vee y) \vee \neg z) \vee \neg ((\neg x \vee \neg y) \vee \neg z)
\end{align*}
Eine Wahrheitstafel hätte mir hier zu viele Spalten, aber das Argument warum die DNF-Formel korrekt ist sollte klar sein und dass DeMorgan und doppelte Verneinung Äquivalenzumformungen sind geht aus den Folien hervor.
\end{document}
